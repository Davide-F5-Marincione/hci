\documentclass[a4paper, 11pt, twocolumn]{article}

\usepackage{microtype} % Improves spacing
\usepackage[utf8]{inputenc} % Required for inputting international characters
\usepackage[T1]{fontenc} % Output font encoding for international characters

\usepackage{kpfonts,baskervald}

\usepackage[scale=.8]{geometry}

\title{Interview Answers}
\author{Dario Loi \and Elena Maria Muià}
\date{2023--04--26}


\begin{document}
\maketitle
\paragraph*{Interview 1 --- 20 y.o.}
\begin{enumerate}
	\item Yes. Metro, bus, train, and tram.
	\item The user chooses mostly the metro because it is faster and there are fewer stopovers.
	\item Overcrowded, delays, cancellations.
	\item Moovit, Google Maps, Probus. The user uses them both to find the right path and timetables.
	\item Enough trusting. Not totally because sometimes the path is not actually the shortest and some existing paths are not even considered.
	\item Not particularly worried. If a classification is needed, he/she is more worried about sharing the mic and camera information.
	\item Not worried.
	\item He/she would not trust a community because people are not always sincere. Would like to participate in case of actual collaboration.
	\item He/she would be incentivized by an honor system, in particular, if it would involve the unlocking of new contents in the application.
	\item Not interested in profile customization.

\end{enumerate}

\paragraph*{Interview 2 --- 20 y.o.}
\begin{enumerate}
	\item Yes. Train and tram.
	\item Quickness, those which brings the user as close as possible to the destination.
	\item Overcrowded, waiting.
	\item Does use an app. To find out the waiting time.
	\item Enough. Sometimes the given information is not truthful.
	\item Not worried about sharing. If a classification is needed, he/she is more worried about sharing photo gallery and location information.
	\item Not much worried about sharing location information with third parties.
	\item Yes, possible if it were both centralized and crowd-sourced. She would participate.
	\item Yes it will.
	\item She is not interested in avatar customization. It would be more interesting in content unlocking useful for him/her.
\end{enumerate}

\paragraph*{Interview 3 --- 19 y.o.}
\begin{enumerate}
	\item Yes. Tram, bus, trains
	\item Quickness.
	\item Timetables are not accurate.
	\item Yes, Moovit seems the most accurate. To find timetables.
	\item $8/10$ trust. Timetables are not ALWAYS on point but he/she is overall trusting.
	\item Gives authorization to share data only when he/she is using the application, but is not very worried. Particularly worried about sharing information about the position $24/7$.
	\item A bit worried because he/she does not understand the purpose of it.
	\item Yes. Yes.
	\item Yes. Both customization and content unlocking.
	\item Yes he is interested.
\end{enumerate}

\paragraph*{Interview 4 --- 19 y.o.}
\begin{enumerate}
	\item Yes. Tram and bus.
	\item Timetables, quickness.
	\item Timetables are never truthful.
	\item Yes. To buy tickets and to find which bus is faster.
	\item Enough. Timetables are not always truthful.
	\item Not worried. He/she is only careful about sharing location only when the application is open.
	\item Not worried about sharing with third parties.
	\item Probably yes. Would prefer a system both centralized and crowd-sourced. But he/she is not trusting about general participation.
	\item Yes. Prefers content unlocking.
	\item Yes, he/she is interested in it.
\end{enumerate}

\paragraph*{Interview 5 --- 21 y.o.}
\begin{enumerate}
	\item Yes. Metro and bus.
	\item Quickness, first to arrive.
	\item Overcrowded, waiting time.
	\item Yes, Moovit mostly. Used to find out timetables.
	\item $8/10$ of trust in Moovit, 0/10 in Google Maps. Sometimes the information given is not correct because they do not follow the departure times and there are delays.
	\item he/she is sometimes worried about giving authorization for data access, mostly when she does not understand why it is needed.
	\item Yes, when she does not understand why it is needed.
	\item Yes. Would prefer a system both centralized and crowd-sourced. Would be a part of it.
	\item Yes. Prefers content unlocking.
	\item No. In particular not in the mobility field.
\end{enumerate}


\paragraph*{Interview 6 --- 21 y.o.}
\begin{enumerate}
	\item Yes. Train, bus, metro.
	\item Expensiveness, quickness.
	\item Yes. To schedule future movements.
	\item The given information is not always truthful.
	\item Enough. Bus info is not accurate, train one is more trustable.
	\item Enough worried. More worried about personal information (e.g., name, age). Will not share information tracked by phone sensors.
	\item Worried, because he/she is giving a piece of personal information without receiving anything in return.
	\item Yes. He/she would even prefer a system fully crowd-sourced and would be a part of it.
	\item Yes. In particular, he/she is interested in discounts, and concrete incentives (including customization). He/she is not interested in content unlocking because the app should be equally and fully working for everyone from the first download.
	\item Non very interested.
\end{enumerate}

\paragraph*{Interview 7 --- 19 y.o.}
\begin{enumerate}
	\item Yes. Train, bus, tram.
	\item Shortest path, fewer stepovers.
	\item Yes. Uses Probus to find the current bus position and waiting time. Uses Google Maps to find the shortest path.
	\item Overcrowded, not enough personal space, people jostling by.
	\item $6/10$ of the trust. With Google Maps, timetables are not truthful, especially during nighttime. Probus only gives information about the current bus but if it does not pass, or if it already passed, there is no information about the next trip.
	\item  Depends on the type of authorization and on how useful it is to give authorization.
	\item Never thought about it.
	\item Yes and he/she would be a part of it if she would be able to. She would even prefer a system totally crowd-sourced.
	\item Yes. Prefers content unlocking over profile customization.
	\item Yes, but he/she does not know how it can be useful for a mobility app.
\end{enumerate}

\paragraph*{Interview 8 --- 23 y.o.}
\begin{enumerate}
	\item Yes. Metro mostly.
	\item Quickness, reliability, comfortability.
	\item Waiting time, traffic jam on the bus.
	\item Yes.
	\item he/she trusts them but sometimes the suggested path is not correct.
	\item Not worried at all about any of them.
	\item Would be worried if location information would be shared with third parties.
	\item Yes. But he/she is too lazy to be an active part of it.
	\item No.
	\item No.
\end{enumerate}


\paragraph*{Interview 9 --- 27 y.o.}

\begin{enumerate}
	\item Yes, trains mostly.
	\item Quickness and departure time.
	\item Not many available trips and some paths require a lot of stepovers.
	\item Yes, rarely. To find the correct path and to check that it is the shortest one.
	\item $10/10$, in particular, he/she trusts a lot Google maps because it is managed very scrupulously.
	\item Not worried. If a classification is needed, he/she is more worried about sharing heading and photo gallery information.
	\item Probably yes.
	\item Maybe at the same level of trust that he/she is feeling with the current system. But he/she would like to add a community and would be part of it.
	\item Yes. He/she would prefer economic incentives but between customization and content unlocking prefers the latter
	\item No.
\end{enumerate}

\paragraph*{Interview 10 --- 22 y.o.}
\begin{enumerate}
	\item Yes. Bus and train.
	\item Quickness, frequency
	\item Frequency.
	\item Yes. To find the correct and shortest path.
	\item $10/10$ even if the bus does not follow the fixed timetables.
	\item Not worried.  If a classification is needed, he/she is more worried about sharing location information.
	\item he/she does not like it very much.
	\item Yes and he/she would be part of it only if the contribution is not too demanding.
	\item Yes but not that much, because he/she feels that her contribution would be driven by mutual utility rather than by personal advantage. Prefers content unlocking over profile customization.
	\item No.
\end{enumerate}
\end{document}
