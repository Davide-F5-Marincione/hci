\documentclass[a4paper, 11pt]{report}


%────────────────────────────────────────────────────────────────────────────────────────────────────────────────────────────────────────────────────
%
%   PACKAGES
%
%────────────────────────────────────────────────────────────────────────────────────────────────────────────────────────────────────────────────────

\usepackage{microtype}
\usepackage{hyperref}
\usepackage[dvipsnames,table]{xcolor}
% remove the outline of the links

\hypersetup{
    colorlinks,
    linkcolor={black},
    citecolor={black},
    urlcolor={black}
}

\usepackage[scale=0.70]{geometry}
\usepackage{graphicx}
\usepackage{lipsum}


\usepackage[utf8]{inputenc} % Required for inputting international characters
\usepackage[T1]{fontenc} % Output font encoding for international characters

\usepackage{kpfonts,baskervald}

\usepackage{parskip}
\usepackage{tcolorbox}

\newtcolorbox{basecolorbox}[1][]{%
  colback=gray!25, colframe=gray!25,
  coltitle=black, fonttitle=\Large\bfseries,
  rounded corners,
  width = (\linewidth-30pt),
  title=#1}

\newenvironment{titlebox}[1][]{%
  \centering
  \basecolorbox[#1]%
}{%
  \endbasecolorbox%
}

\newenvironment{inlinebox}[1][]{%
  \titlebox%
  {\upshape\bfseries #1}%
}{%
  \endbasecolorbox%
}


\usepackage{cleveref}

\author{Martina Doku \and Giuseppina Iannotti \and Dennis Locatelli \and Elena Muià \and Dario Loi \and Davide Marincione}

\title{\Huge Report for HCI} % to be changed

\begin{document}

\begin{titlepage} % Suppresses displaying the page number on the title page and the subsequent page counts as page 1
	\newcommand{\HRule}{\rule{\linewidth}{0.5mm}} % Defines a new command for horizontal lines, change thickness here

	\center % Centre everything on the page

	%------------------------------------------------
	%   Headings
	%------------------------------------------------

	\textsc{\LARGE Sapienza University of Rome}\\[1.5cm] % Main heading such as the name of your university/college

	\textsc{\Large Human Computer Interaction -- A.Y 2022/2023}\\[0.5cm] % Major heading such as course name

	\textsc{\large Applied Computer Science and Artificial Intelligence}\\[0.5cm] % Minor heading such as course title

	%------------------------------------------------
	%   Title
	%------------------------------------------------

	\HRule\\[0.4cm]

	{\huge\bfseries A Crowd-Sourced Mobility Application}\\[0.4cm] % Title of your document

	\HRule\\[1.5cm]

	%------------------------------------------------
	%   Author(s)
	%------------------------------------------------

	\begin{minipage}{0.4\textwidth}
		\begin{flushleft}
			\Large
			\textit{Authors:}\\[3pt]\hspace{2pt} % insert a 2pt indentation
			Doku \textsc{Martina}\\[1pt]\hspace{2pt}
			Iannotti \textsc{Giuseppina}\\[1pt]\hspace{2pt}
			Locatelli \textsc{Dennis}\\[1pt]\hspace{2pt}
			Muià \textsc{Elena Maria}\\[1pt]\hspace{2pt}
			Loi \textsc{Dario}\\[1pt]\hspace{2pt}
			Marincione \textsc{Davide}
		\end{flushleft}
	\end{minipage}
	\begin{minipage}{0.4\textwidth}
		\begin{flushright}
		\end{flushright}
	\end{minipage}

	% If you don't want a supervisor, uncomment the two lines below and comment the code above
	%{\large\textit{Author}}\\
	%John \textsc{Smith} % Your name

	%------------------------------------------------
	%   Date
	%------------------------------------------------

	\vfill\vfill\vfill % Position the date 3/4 down the remaining page

	{\large May 2, 2023} % Date, change the \today to a set date if you want to be precise

	%------------------------------------------------
	%   Logo
	%------------------------------------------------

	%\vfill\vfill
	%\includegraphics[width=0.2\textwidth]{placeholder.jpg}\\[1cm] % Include a department/university logo - this will require the graphicx package

	%----------------------------------------------------------------------------------------

	\vfill % Push the date up 1/4 of the remaining page

\end{titlepage}

\tableofcontents

\chapter{Introduction}
%Decide if we want to scale back to section, find a way to
% suppress the numbering of the chapter so that first section is
% 1.0 and not 0.1
\section{Our Idea}\label{sec:our-idea}
The intent of our application is to give more precise information about possible delays or irregularities in the public transportation service. This could be achieved by the retrieval of information delivered directly by the community of users, who may have the chance to inform others about possible delays or overcrowding of the bus or the tram that they are taking.
Furthermore, all the users will collect some credits for every contribution, which will be devolved into charity, in one of the NGOs chosen by the user, among those proposed by the application.

\section{Existing Competitors}\label{sec:existing-competitors}

\paragraph{Google Maps:} It is one of Italy's most frequently used mobility applications. It is a web service that provides detailed information about geographical regions and sites worldwide. In addition to conventional road maps, Google Maps offers aerial and satellite views of many locations. It works both for public transportation and private ones.
\begin{itemize}
	\item \textbf{Pros:} It is very intuitive and offers many fundamental features. It delivers information about possible paths to reach the destination with their duration and arrival time. It specifies the possible expenses needed for every choice taken.
	\item \textbf{Cons:} The path duration is not defined by crowdsourced information or GPS tracking applications. It is based totally on statistical inferences which means that is almost never precise.
\end{itemize}

\paragraph{Moovit:}  It is one of the main mobility applications for public transportation only. It provides information about the statistically best path to reach a destination point. Describing when and where to take the transportation means in order to reach the destination as fast as possible.

\begin{itemize}
	\item \textbf{Pros:} More reliable than Google Maps (based on users' interviews), the interface is very intuitive.
	\item \textbf{Cons:} Too many ads, the defined timing is almost never correct.
\end{itemize}

\paragraph{Probus:}The application is designed for Android only and it only works with buses. It informs the user about the waiting time of a certain bus line and the fastest path to reach a destination.

\begin{itemize}
	\item \textbf{Pros:} Useful because it is focused only on the waiting time of the bus.
	\item \textbf{Cons:} The client service is not very reliable. It is very much reliable with the current bus trip but it does not give information about future ones so the user does not know how long he/she will have to wait for the next run.
\end{itemize}

\paragraph{Citymapper:} Citymapper is a public transit app and mapping service which displays transport options, usually with live timing, between any two locations in a supported city. It integrates data for all urban modes of transport, including walking, cycling, and driving, in addition to public transport.

\begin{itemize}
	\item \textbf{Pros:} Almost always accurate, comprehensive direction guide, free for both Android and IoS, it provides a calories counter and 	specifies the expenses for every chosen path.
	\item \textbf{Cons:} Not available in many cities and it does not retrive crowdsourced information.
\end{itemize}

\paragraph{Transit:}  Transit is a mobile app packed with features that helps you plan a trip on The Bus. Real-time bus tracking and information, service alerts, and trip planners are some of the many useful features that make this app the favorite for transportation services.


\begin{itemize}
	\item \textbf{Pros:} GPS tracking of public transportation in real-time, crowdsource support (tracking the user location when they use the app as a navigator), information about all the surrounding bus stops and possible paths to the destination.
	\item \textbf{Cons:} Many useful services are not free, it does not work very well in Italy.
\end{itemize}

\section{Need Finding}\label{sec:need-finding}

\paragraph{The Interviews} 

In order to better understand what our users want, we first conducted a round of interviews,
these allowed us to interact colloquially with our potential users and to gauge what they think 
are the major discomforts of public transportation. We also wanted to understand their approach 
to personal privacy and community-driven applications. We used this data as a guide for our 
next steps in the design process.

\subsection{Our Questions}\label{subsec:our-questions}

Our interviews were standardized around a set of ten questions that we designed, as a group, 
to be as open-ended as possible. We wanted to avoid leading the interviewees to answer in a
particular way, have them act as designers, or figure out the specific purpose of the 
survey until later on, when the general questions were answered.\\[2pt]

\begin{titlebox}[The Questions:]
	\begin{enumerate}
		\item Did you commute via public transport in the last week? If so, what type?
		\item What criteria do you consider when choosing your means of transportation?
		\item What are some frustrating aspects about public transportation? 
		\item Do you use mobility apps (like Google Maps) while commuting? If so, which
		functionalities? 
		\item How much do you trust the information given by your app of choice? 
		\item Do you worry about giving authorizations to apps? Are there some you are more willing to share? 
		\item Are you concerned about organizations distributing your location based data to third parties? 
		\item Would you trust mobility info more if it were crowd-sourced? Would you participate 
		in such a program? 
		\item Would a honor system, rewarding you based on the credibility of
		your contributions, incentivize you to participate more? 
		\item In a community-driven app, how interested are you in customizing and showing your
		profile?
	\end{enumerate}
\end{titlebox}

\paragraph{The Outcomes}

To carry out the interviews, we split our groups into 3 teams of 2 people each. Each of those 
teams had a target of 10 interviews to reach, which was achieved in a few days (with 
some extra interviews to spare).


\end{document}
